\PassOptionsToPackage{unicode=true}{hyperref} % options for packages loaded elsewhere
\PassOptionsToPackage{hyphens}{url}
%
\documentclass[]{article}
\usepackage{lmodern}
\usepackage{amssymb,amsmath}
\usepackage{ifxetex,ifluatex}
\usepackage{fixltx2e} % provides \textsubscript
\ifnum 0\ifxetex 1\fi\ifluatex 1\fi=0 % if pdftex
  \usepackage[T1]{fontenc}
  \usepackage[utf8]{inputenc}
  \usepackage{textcomp} % provides euro and other symbols
\else % if luatex or xelatex
  \usepackage{unicode-math}
  \defaultfontfeatures{Ligatures=TeX,Scale=MatchLowercase}
\fi
% use upquote if available, for straight quotes in verbatim environments
\IfFileExists{upquote.sty}{\usepackage{upquote}}{}
% use microtype if available
\IfFileExists{microtype.sty}{%
\usepackage[]{microtype}
\UseMicrotypeSet[protrusion]{basicmath} % disable protrusion for tt fonts
}{}
\IfFileExists{parskip.sty}{%
\usepackage{parskip}
}{% else
\setlength{\parindent}{0pt}
\setlength{\parskip}{6pt plus 2pt minus 1pt}
}
\usepackage{hyperref}
\hypersetup{
            pdftitle={Database interoperability for spatial objects in R},
            pdfauthor={Etienne Racine and Edzer Pebesma},
            pdfborder={0 0 0},
            breaklinks=true}
\urlstyle{same}  % don't use monospace font for urls
\usepackage[margin=1in]{geometry}
\usepackage{color}
\usepackage{fancyvrb}
\newcommand{\VerbBar}{|}
\newcommand{\VERB}{\Verb[commandchars=\\\{\}]}
\DefineVerbatimEnvironment{Highlighting}{Verbatim}{commandchars=\\\{\}}
% Add ',fontsize=\small' for more characters per line
\usepackage{framed}
\definecolor{shadecolor}{RGB}{248,248,248}
\newenvironment{Shaded}{\begin{snugshade}}{\end{snugshade}}
\newcommand{\AlertTok}[1]{\textcolor[rgb]{0.94,0.16,0.16}{#1}}
\newcommand{\AnnotationTok}[1]{\textcolor[rgb]{0.56,0.35,0.01}{\textbf{\textit{#1}}}}
\newcommand{\AttributeTok}[1]{\textcolor[rgb]{0.77,0.63,0.00}{#1}}
\newcommand{\BaseNTok}[1]{\textcolor[rgb]{0.00,0.00,0.81}{#1}}
\newcommand{\BuiltInTok}[1]{#1}
\newcommand{\CharTok}[1]{\textcolor[rgb]{0.31,0.60,0.02}{#1}}
\newcommand{\CommentTok}[1]{\textcolor[rgb]{0.56,0.35,0.01}{\textit{#1}}}
\newcommand{\CommentVarTok}[1]{\textcolor[rgb]{0.56,0.35,0.01}{\textbf{\textit{#1}}}}
\newcommand{\ConstantTok}[1]{\textcolor[rgb]{0.00,0.00,0.00}{#1}}
\newcommand{\ControlFlowTok}[1]{\textcolor[rgb]{0.13,0.29,0.53}{\textbf{#1}}}
\newcommand{\DataTypeTok}[1]{\textcolor[rgb]{0.13,0.29,0.53}{#1}}
\newcommand{\DecValTok}[1]{\textcolor[rgb]{0.00,0.00,0.81}{#1}}
\newcommand{\DocumentationTok}[1]{\textcolor[rgb]{0.56,0.35,0.01}{\textbf{\textit{#1}}}}
\newcommand{\ErrorTok}[1]{\textcolor[rgb]{0.64,0.00,0.00}{\textbf{#1}}}
\newcommand{\ExtensionTok}[1]{#1}
\newcommand{\FloatTok}[1]{\textcolor[rgb]{0.00,0.00,0.81}{#1}}
\newcommand{\FunctionTok}[1]{\textcolor[rgb]{0.00,0.00,0.00}{#1}}
\newcommand{\ImportTok}[1]{#1}
\newcommand{\InformationTok}[1]{\textcolor[rgb]{0.56,0.35,0.01}{\textbf{\textit{#1}}}}
\newcommand{\KeywordTok}[1]{\textcolor[rgb]{0.13,0.29,0.53}{\textbf{#1}}}
\newcommand{\NormalTok}[1]{#1}
\newcommand{\OperatorTok}[1]{\textcolor[rgb]{0.81,0.36,0.00}{\textbf{#1}}}
\newcommand{\OtherTok}[1]{\textcolor[rgb]{0.56,0.35,0.01}{#1}}
\newcommand{\PreprocessorTok}[1]{\textcolor[rgb]{0.56,0.35,0.01}{\textit{#1}}}
\newcommand{\RegionMarkerTok}[1]{#1}
\newcommand{\SpecialCharTok}[1]{\textcolor[rgb]{0.00,0.00,0.00}{#1}}
\newcommand{\SpecialStringTok}[1]{\textcolor[rgb]{0.31,0.60,0.02}{#1}}
\newcommand{\StringTok}[1]{\textcolor[rgb]{0.31,0.60,0.02}{#1}}
\newcommand{\VariableTok}[1]{\textcolor[rgb]{0.00,0.00,0.00}{#1}}
\newcommand{\VerbatimStringTok}[1]{\textcolor[rgb]{0.31,0.60,0.02}{#1}}
\newcommand{\WarningTok}[1]{\textcolor[rgb]{0.56,0.35,0.01}{\textbf{\textit{#1}}}}
\usepackage{graphicx,grffile}
\makeatletter
\def\maxwidth{\ifdim\Gin@nat@width>\linewidth\linewidth\else\Gin@nat@width\fi}
\def\maxheight{\ifdim\Gin@nat@height>\textheight\textheight\else\Gin@nat@height\fi}
\makeatother
% Scale images if necessary, so that they will not overflow the page
% margins by default, and it is still possible to overwrite the defaults
% using explicit options in \includegraphics[width, height, ...]{}
\setkeys{Gin}{width=\maxwidth,height=\maxheight,keepaspectratio}
\setlength{\emergencystretch}{3em}  % prevent overfull lines
\providecommand{\tightlist}{%
  \setlength{\itemsep}{0pt}\setlength{\parskip}{0pt}}
\setcounter{secnumdepth}{0}
% Redefines (sub)paragraphs to behave more like sections
\ifx\paragraph\undefined\else
\let\oldparagraph\paragraph
\renewcommand{\paragraph}[1]{\oldparagraph{#1}\mbox{}}
\fi
\ifx\subparagraph\undefined\else
\let\oldsubparagraph\subparagraph
\renewcommand{\subparagraph}[1]{\oldsubparagraph{#1}\mbox{}}
\fi

% set default figure placement to htbp
\makeatletter
\def\fps@figure{htbp}
\makeatother


\title{Database interoperability for spatial objects in R}
\author{Etienne Racine and Edzer Pebesma}
\date{2020-04-01}

\begin{document}
\maketitle

\hypertarget{signatories}{%
\section{Signatories}\label{signatories}}

\hypertarget{project-team}{%
\subsection{Project team}\label{project-team}}

Etienne Racine, Edzer Pebesma

\hypertarget{contributors}{%
\subsection{Contributors}\label{contributors}}

\hypertarget{consulted}{%
\subsection{Consulted}\label{consulted}}

\hypertarget{the-problem}{%
\section{The Problem}\label{the-problem}}

Large spatial data can be hard to analyze with R because of RAM
limitations. Users often turn to spatial databases for this process and
go back-and-forth with R, since R is still needed for e.g.~modeling.
Also corporations and large research groups, among others, often store
data in a central database. The ability to interact directly with a
spatial database, from R, can accelerate the work of R users.

The sf package is built on the simple feature OGC standard, which
facilitates the interoperability with spatial databases. We already
provide read and write operations to postgis from sf, and we can also
integrate with the dbplyr package to push execution into the database.
The gdal driver within sf also provides interoperability, but does not
allow integration with the dplyr workflow.

This proposal is to :

\begin{itemize}
\tightlist
\item
  Complete and clarify the interface for the \texttt{RPostgres} and
  \texttt{odbc} driver
\item
  Create a tutorial to use the database interface
\end{itemize}

\hypertarget{the-proposal}{%
\section{The proposal}\label{the-proposal}}

\begin{itemize}
\tightlist
\item
  Clarify the workflow integration with dplyr
\item
  How to select a database spatial engine (maybe we could wrap the
  connection with a spatial extension)
\item
  Merge work form geotidy to sf
\item
  Document the workflow
\item
  Provide tutorials
\item
  Complete odbc support for Postgres, which would potentially open to
  other spatial extensions
\item
  Document the features needed for DBI drivers (\texttt{SQLite},
  \texttt{MariaDB}, \texttt{bigrquery})
\end{itemize}

\hypertarget{overview}{%
\subsection{Overview}\label{overview}}

This proposal builds on the successfully funded simple feature
(\texttt{sf}) project. \texttt{sf} is built on the simple feature OGC
standard, which facilitates the interoperability with spatial databases.
The sf package already supports read and write for postgis, one of the
most common and powerful spatial databases, and can use the dbplyr
package to push execution into the database. The gdal driver also
provides interoperability, but does not allow integration with the dplyr
workflow. This feature is however is not very well documented, and
little used.

The proposal is to complete the database support by making more sf
operations compatible with other databases, facilitate integration tests
by moving database support to another package (sfdbi) and add support
for other databases through odbc, potentially unlocking many other
spatial databases compatible with the OGC standard. The following table
provides some of the most important spatial databases in use.

\begin{verbatim}
## # A tibble: 17 x 4
##    Database             `Spatial extension` `R Driver`  `Current support☨`           
##    <chr>                <chr>               <chr>       <chr>                        
##  1 PostgreSQL           Postgis             Postgres    Partial                      
##  2 PostgreSQL           Postgis             RPostgreSQL Partial                      
##  3 PostgreSQL           Postgis             odbc        Partial - Not tested         
##  4 Couchdb              Geocouch            sofa        Unknown                      
##  5 PostgreSQL           ArcSDE              odbc        Unknown - Reported usage     
##  6 Spark                Geospark            sparklyr    Partial with geospark package
##  7 Oracle               Oracle Spatial      odbc        Unknown                      
##  8 Microsoft SQL Server Native              odbc        Unknown - Reported usage     
##  9 MySQL                Native (primitive)  odbc        Unknown                      
## 10 SQlite               SpatialLite         RSQLite     Unknown                      
## 11 MariaDB              Native              RMariaDB    Unknown                      
## 12 BigQuery             BigQuery GIS        bigrquery   Unknown                      
## 13 H2                   H2GIS               RH2         Unknown                      
## 14 Redshift             Native (primitive)  odbc        Unknown                      
## 15 MongoDB              Native (primitive)  mongolite   Unknown                      
## 16 Parquet              None                arrow       Unknown - Speculative        
## 17 Feather              None                arrow       Unknown - Speculative
\end{verbatim}

☨ Partial support: is currently supported by sf, but the set of features
is limited; Not tested: not included in the continuous integration tests
(although there are some ad-hoc tests that can be run manually);
Unknown: not tested, might work partially; Reported usage: not tested,
but users have notified they had used it; Speculative: we've had
discussions with the Ursa Labs team, and it seems possible and will
require development on our side and on the arrow side -- this would
greatly facilitate interoperability with the spatial python ecosystem
and geopandas.

\hypertarget{detail}{%
\subsection{Detail}\label{detail}}

Minimally, we propose to support dplyr workflows with collect, tbl,
copy\_to et al.~methods for postgis\_connection objects. Expand sf
functions to be compatible with postgis to the extent possible Teach,
through tutorials how to interface with a spatial database and

The plan is: In sf, refactor code so we can extract the database
interface in a separate package It will make maintenance easier
Accelerate integration tests Remove DBI dependency In sf, adapt, to the
extent possible, functionalities for the database interaction Build a
second package, sfdbi, to contain the spatial interface The intended
design is to create a spatial database class that could be used to
extend the tbl functionality to read and write spatial data using
copy\_to() and collect(). This is what a minimal usage would look like:

\begin{Shaded}
\begin{Highlighting}[]
\CommentTok{# A DBI connection modifier (to allow support for more }
\CommentTok{# spatial databases (see table 1)}
\NormalTok{postgis <-}\StringTok{ }\ControlFlowTok{function}\NormalTok{(...)\{}
    \KeywordTok{structure}\NormalTok{(..., }\DataTypeTok{class =} \KeywordTok{c}\NormalTok{(}\StringTok{"postgis_connection"}\NormalTok{, }\KeywordTok{class}\NormalTok{(...)))}
\NormalTok{\}}

\CommentTok{# Usage}
\NormalTok{con <-}\StringTok{ }\NormalTok{DBI}\OperatorTok{::}\KeywordTok{dbConnect}\NormalTok{(RPostgres}\OperatorTok{::}\KeywordTok{Postgres}\NormalTok{()) }\OperatorTok
\StringTok{    }\KeywordTok{postgis}\NormalTok{()}

\CommentTok{# Copy spatial data to database}
\NormalTok{x <-}\StringTok{ }\KeywordTok{copy_to}\NormalTok{(con, }\KeywordTok{tibble}\NormalTok{(}
    \DataTypeTok{lon =} \DecValTok{1}\OperatorTok{:}\DecValTok{3}\NormalTok{,}
    \DataTypeTok{lat =} \DecValTok{3}\OperatorTok{:}\DecValTok{1}\NormalTok{,}
    \DataTypeTok{geom =} \KeywordTok{st_makepoint}\NormalTok{(longitude, latitude) }\OperatorTok\StringTok{ }\KeywordTok{st_setsrid}\NormalTok{(}\DecValTok{4326}\NormalTok{)}
\NormalTok{))}

\KeywordTok{collect}\NormalTok{(x)  }\CommentTok{# would return a tbl with an sfc column}
\end{Highlighting}
\end{Shaded}

\hypertarget{project-plan}{%
\section{Project plan}\label{project-plan}}

\hypertarget{start-up-phase}{%
\subsection{Start-up phase}\label{start-up-phase}}

\hypertarget{technical-delivery}{%
\subsection{Technical delivery}\label{technical-delivery}}

\hypertarget{other-aspects}{%
\subsection{Other aspects}\label{other-aspects}}

\hypertarget{requirements}{%
\section{Requirements}\label{requirements}}

The solution is clear, and the implementation seems to be relatively
simple. in the worst case, we could minimize the changes to \texttt{sf}
to avoid breaking existing code. We don't expect to require changes to
\texttt{dbplyr}, but this could a source of delays or modification to
the plan. \texttt{rodbc} could also need some modifications, but we have
already existing workarounds. We have not identified other external
factors that could impact this project.

\hypertarget{people}{%
\subsection{People}\label{people}}

Etienne will lead the project: refactor, develop and document; Edzer
will be involved in design and approval for \texttt{sf}.

\hypertarget{processes}{%
\subsection{Processes}\label{processes}}

The new package \texttt{sfdbi} will be added to the r-spatial.org and
will adhere to the code of conduct of the The tutorials will be shared
via github pages and ideally could be run on docker, rstudio.cloud, or
netlify to make it easy for the community to use.

\hypertarget{tools-tech}{%
\subsection{Tools \& Tech}\label{tools-tech}}

We will use github actions to test on linux versions (currently the only
OS to support services on github actions).

\hypertarget{funding}{%
\subsection{Funding}\label{funding}}

We request US\$ 6,000 for this project.

\hypertarget{summary}{%
\subsection{Summary}\label{summary}}

\begin{itemize}
\tightlist
\item
  Refactor \texttt{sf} (2000)
\item
  Develop \texttt{sfdbi} (2000)
\item
  Write a tutorial and polish documentation (2000)
\end{itemize}

\hypertarget{success}{%
\section{Success}\label{success}}

\hypertarget{definition-of-done}{%
\subsection{Definition of done}\label{definition-of-done}}

We can interact with a \texttt{PostgreSQL} database from R by using
\texttt{dplyr} workflow and people use it because the documentation is
clear and there are good tutorials.

\hypertarget{measuring-success}{%
\subsection{Measuring success}\label{measuring-success}}

Increased number of users of the spatial database interface.

\hypertarget{future-work}{%
\subsection{Future work}\label{future-work}}

This work hopefully makes it easier to add support for other spatial
databases (table 1).

\hypertarget{key-risks}{%
\subsection{Key risks}\label{key-risks}}

\begin{itemize}
\tightlist
\item
  Pandemic
\end{itemize}

\end{document}
